\documentclass{article}
\usepackage[a4paper]{geometry}
\usepackage{booktabs}
\usepackage{cite}
\usepackage{longtable}
\usepackage{verbatim}
\usepackage{graphicx}
\usepackage{amsmath}
\usepackage{bm}
\usepackage{lscape}
\usepackage{pdfpages}
\usepackage{tikz}
\usepackage{hyperref}
\usepackage{parskip}
\usepackage{multirow}
\usepackage{color}

\author{camilla.berge.vik@ntnu.no}
\title{Two-Phase Plug-Flow Model for Fischer-Tropsch Synthesis}

\begin{document}

\section{Symbol List}
\begin{center}
	\begin{longtable}{lll}
	\caption{Symbol list} \\
	\label{tab:symbol_list}
		\textbf{symbol} & \textbf{name} & \textbf{unit} \\
		\endhead
			$k_{L,i}$ & Liquid side mass transfer coefficient & $m\,s^{-1}$ \\
			$a_L$ & Gas-liquid interfacial area per unit liquid volume & $m^2/m^3= m^{-1}$ \\
			$k_{L,i}a$ & Liquid side volumetric mass transfer coefficient &  $m^{-1}\,m\,s^{-1}=s^{-1}$ \\
			$y_i$ & Gas phase mole fraction of component $i$& $-$ \\
			$x_i$ & Liquid phase mole fraction of component $i$& $-$ \\
			$y_i^*$ & Gas phase mole fraction of component $i$ at the phase boundary & $-$ \\
			$x_i^*$ & Liquid phase mole fraction of component $i$ at the phase boundary & $-$ \\
			$k_i$ & "Constant" describing the relationship $y_i^*/x_i^*$ at {\color{red}given conditions} & $-$ \\
			$\omega_{i,G}$ & Gas phase weight fraction of component $i$& $-$ \\
			$\omega_{i,G}^*$ & Gas phase weight fraction of component $i$ at the phase boundary & $-$ \\
			$\omega_{i,L}$ & Liquid phase weight fraction of component $i$& $-$ \\
			$\omega_{i,L}^*$ & Liquid phase weight fraction of component $i$ at the phase boundary& $-$ \\
			$z$ & Reactor height & $m$ \\
			$\rho_L$ & Total density, liquid phase & $kg\,m^{-3}$ \\
			$\rho_G$ & Total density, gas phase & $kg\,m^{-3}$ \\
			$\rho_S=\rho_{\mathrm{cat}}$ & Catalyst density density & $kg\,m^{-3}$ \\
			$\rho_{SL}$ & Slurry density & $kg\,m^{-3}$ \\
			$v_G$ & Gas velocity & $m\,s^{-1}$\\
			$v_L$ & Liquid velocity & $m\,s^{-1}$\\
			$v_{SL}$ & Slurry velocity & $m\,s^{-1}$\\
			$v_G^s$ & Gas superficial velocity & $m\,s^{-1}$\\
			$v_L^s$ & Liquid superficial velocity & $m\,s^{-1}$\\
			$v_{SL}^s$ & Slurry superficial velocity & $m\,s^{-1}$\\
			$\overline{M_G}$ & Gas phase average molar mass & $kg\,kmol^{-1}$\\
			$\overline{M_L}$ & Liquid phase average molar mass & $kg\,kmol^{-1}$\\
			$\epsilon_L$ & Liquid volume fraction & $-$\\
			$\epsilon_G$ & Gas volume fraction & $-$\\
			$\epsilon_S$ & Solid volume fraction & $-$\\
			$\epsilon_{SL}$ & Slurry volume fraction & $-$\\
			$\phi_L$ & Liquid mass fraction & $-$\\
			$\phi_G$ & Gas mass fraction & $-$\\
			$\phi_S$ & Solid mass fraction & $-$\\
			$\phi_{SL}$ & Slurry mass fraction & $-$\\
			$\nu_i$ & Stochiometric coefficient for component $i$ & $-$ \\
			$r_{CO}$ & Reaction rate in terms of $CO$ conversion & $kmol\, s^{-1}\,kg_{cat}^{-1}$\\
			$M_{w,i}$ & Molecular weight of component $i$ & $kg\,kmol^{-1}$\\
			$\rho_{cat}$ & Density of catalyst & $kg_{cat}\,m^{-3}$\\ 
			$T$ & Temperature & $K$\\
			$T_G$ & Temperature, Gas Phase & $K$\\
			$T_L$ & Temperature, Liquid Phase & $K$\\
			$T_{SL}$ & Temperature, Slurry Phase & $K$\\
			$T_{\mathrm{surr}}$ & Temperature, Surroundings & $K$ \\
			$p$ & Pressure & $Pa$ \\
			$p_G$ & Gas phase pressure & $Pa$ \\
			$p_L$ & Liquid phase pressure & $Pa$ \\
			$h$ & Height of dispersion & $m$ \\ 
			$R$ & Gas constant (value: $8.314\times10^{3}$) & $J\,K^{-1}\,kmol^{-1}$\\
			$L$ & Total height of dispersion & $m$ \\
			$g$ & Standard acceleration of gravity & $ms^{-2}$\\
			$D_{\mathrm{eff,z,G}}$ & Eff. axial dispersion coefficient, gas & $m^2s^{-1}$\\
			$D_{\mathrm{eff,z,L}}$ & Eff. axial dispersion coefficient, liquid & $m^2s^{-1}$\\
			$C_{p,G}$ & Spec. heat (at constant pressure), Gas & $J kg^{-1} K^{-1}$ \\
			$C_{p,S}$ & Spec. heat (at constant pressure), Solid & $J kg^{-1}  K^{-1}$\\
			$C_{p,L}$ & Spec. heat (at constant pressure), Liquid & $J kg^{-1}  K^{-1}$\\
			$C_{p,SL}$ & Spec. heat (at constant pressure), Slurry & $J kg^{-1}  K^{-1}$\\
			$U$ & Thermal Transmittance Number & $Wm^{-2}K^{-1}$ \\
			$d_t$ & Diameter of tube & $m$ \\
			$h_L$ & Heat transfer coefficient between slurry and gas & $W m^{-2}K^{-1}$ \\
			$a$ & Interfacial area density & $m^2m^{-3}=m^{-1}$ \\
			$\lambda_{\mathrm{eff,z,G}}$ & Eff. thermal conductivity, gas & $W m^{-1}K^{-1}$ \\
			$\lambda_{\mathrm{eff,z,L}}$ & Eff. thermal conductivity, liquid & $W m^{-1}K^{-1}$ \\
			$\lambda_{\mathrm{eff,z,S}}$ & Eff. thermal conductivity, solid & $W m^{-1}K^{-1}$ \\
			$\lambda_{\mathrm{eff,z,SL}}$ & Eff. thermal conductivity, slurry & $W m^{-1}K^{-1}$ \\
			$\lambda_G$ & Thermal conductivity, gas & $W m^{-1}K^{-1}$ \\ 
			$\lambda_L$ & Thermal conductivity, liquid & $W m^{-1}K^{-1}$ \\
			$\lambda_{SL}$ & Thermal conductivity, slurry & $W m^{-1}K^{-1}$ \\
			$\mu_G$ & Viscosity, gas & $Pa\,s=kg s^{-1} m^{-1}$ \\
			$\mu_L$ & Viscosity, liquid & $Pa\,s=kg s^{-1} m^{-1}$ \\
			$\mu_{SL}$ & Viscosity, slurry & $Pa\,s=kg s^{-1} m^{-1}$ \\
			$\Delta H_r$ & Reaction heat & $J kmol^{-1}$ \\
			$h_W$ & Heat transfer coefficient between slurry and wall & $W m^{-2}K^{-1}$ \\
			$S_I$ & Perimeter & $m$ \\
			$S_c$ & Perimeter of column & $m$\\
			$S_t$ & Perimeter of cooling tube & $m$\\
			$A$ & Area & $m^2$ \\
			$D_c$ & Diameter of column & $m$ \\
			$A_t$ & Area of cooling tube & $m^2$ \\
			$N_t$ & Number of cooling tubes & $-$ \\
			$d_s^*$ & Sauter mean diameter & $m$ \\
	\end{longtable}
\end{center}

\begin{table}[tb]
	\caption{Subscript and Superscript list}
	\label{tab:subscripts_and_superscripts}
	\begin{center}
		\begin{tabular}{ll}
		\hline

		\hline
		\textbf{symbol} & \textbf{name}\\
		\hline	
			$^*$ & At the phase boundary  \\
			$^0$ & Initial (at the inlet) \\
			$_i$ & Component number (or name) \\
			$^s$ & Superficial \\
			$_c$ & Column \\
			$_t$ & Cooling tube \\
			$_p$ & Constant pressure \\
			$_w$ & Column wall \\
			$_G$ & Gas Side / Phase \\
			$_L$ & Liquid Side / Phase \\
			$_S$ & Solid (catalyst) Side / Phase \\
			$_{SL}$ & Slurry Phase (liquid + catalyst) \\
			$_{SK}$ & Catalyst skeleton (material excluding pores) \\
			$_z$ & Along the $z$ direction \\	
			$_{\mathrm{cat}}$ & Catalyst \\
			$_{\mathrm{eff}}$ & Effective (sum of molecular and turbulent contributions)\\
		\hline

		\hline
		\end{tabular}
	\end{center}
\end{table}
\clearpage

\section{Starting point}
The general axial dispersion model (ADM) for two-phase flow, with reaction in the liquid (slurry) phase and heat exchange in the liquid (slurry) phase, is taken from Jakobsen (2008)~\cite{Jakobsen08}, section 8.4, page 767, equations 8.1 to 8.5. Equation~\ref{eq:total_mass_balance_liquid} (liquid (slurry) phase mass balance), is added.

\subsection{Gas Phase Equations}
\begin{equation}
	\frac{d}{dz}\left(\rho_G v_G^s \omega_{i,G} \right) = \frac{d}{dz}\left(\epsilon_G\rho_GD_{\mathrm{eff,z,G}}\frac{d\omega_{i,G}}{dz} \right) - k_{L,i}a\rho_L(\omega_{L,i}^*-\omega_{L,i}) 
	\label{eq:component_balance_gas}
\end{equation}
\begin{equation}	
	\rho_GC_{p,G}v_G^s\frac{dT_G}{dz} = \frac{d}{dz}\left(\epsilon_G\lambda_{\mathrm{eff,z,G}}\frac{dT_G}{dz}\right) - h_La(T_G-T_{SL})
	\label{eq:energy_balance_gas}
\end{equation}
\begin{equation}
	\frac{d}{dz}\left(\rho_Gv_G^s\right) =-\sum_ik_{L,i}a\rho_L(\omega_{L,i}^*-\omega_{L,i})
	\label{eq:total_mass_balance_gas}
\end{equation}

\subsection{Liquid Phase Equations}
\begin{equation}
	\frac{d}{dz}\left(\rho_L v_L^s \omega_{i,L} \right) = \frac{d}{dz}\left(\epsilon_L\rho_LD_{\mathrm{eff,z,L}}\frac{d\omega_{i,L}}{dz} \right) + k_{L,i}a\rho_L(\omega_{L,i}^*-\omega_{L,i}) + \epsilon_L \sum_r\nu_{i,r}R_{L,r} 
	\label{eq:component_balance_liquid}
\end{equation}
\begin{equation}	
	\rho_LC_{p,L}v_L^s\frac{dT_L}{dz} = \frac{d}{dz}\left(\epsilon_Lk_{\mathrm{eff,z,L}}\frac{dT_L}{dz}\right) + 4 \frac{U}{d_t}(T_{\mathrm{sur}}-T_L) + h_La(T_G-T_L) + \epsilon_L\sum_r(-\Delta H_r)R_{r,L}
	\label{eq:energy_balance_liquid}
\end{equation}
\begin{equation}
	\frac{d}{dz}\left(\rho_Lv_L^s\right) =\sum_ik_{L,i}a\rho_L(\omega_{L,i}^*-\omega_{L,i})
	\label{eq:total_mass_balance_liquid}
\end{equation}

\subsection{Boundary Conditions}
Boundary conditions are given in equations 8.6 to 8.9 in Jakobsen (2008)~\cite{Jakobsen08}. 

\section{Assumptions}
\label{sec:assumptions}
\begin{itemize}
	\item Perfectly mixed slurry phase consisting of liquid with suspended catalyst particles.
	\item Constant volume fraction of solids, $\epsilon_S$.
	\item Only the slurry phase exchanges heat with the wall.
  	\item The pressure is the same in the gas and in the slurry phase; $p_G=p_L$, and follows a hydrostatic profile along the reactor height.
  	\item Ideal gas law valid for gas phase, $pV = nRT$.
  	\item Reaction takes place in the liquid.
  	\item Neglecting mass transfer resistance in gas film, i.e. $\rho_{G,i}^*\approx \rho_{G,i}$
  	\item Equilibrium at the phase boundary between gas and liquid; $y_i^* = k_ix_i ^*$
  	\item $k_{L,i}a(\rho_{L,i}^*-\rho_{L,i}) = k_{L,i}a\rho_L(\omega_{i,L}^*-\omega_{i,L})$
  	\item Assume mass transfer resistance across liquid film surrounding catalyst pellet, and resistance inside pellet, much less than mass transfer resistance across phase boundary (GL). 
  	\item Constant liquid density, $\frac{d \rho_L}{dz} \approx 0$.
  	\item Constant axial dispersion coefficient in both phases, $D_{\mathrm{eff,z,L}}$ and $D_{\mathrm{eff,z,G}}$.
  	\item Constant liquid velocity $v_L=\frac{v_L^s}{\epsilon_L}$ (but \textbf{not} constant superficial velocity $v_L^s$, i.e. $\epsilon_L$ and $v_L^s$ are changing, but $v_L$ is constant).
  	\item Constant thermal conductivity along $z$ direction in both gas and slurry, $\lambda_{\mathrm{eff,z,G}}$ and $k_{\mathrm{eff,z,SL}}$
  \end{itemize}  

\section{Kinetic Model}
\label{sec:kinetic_model}
The kinetic model by Yates and Satterfield (1991)~\cite{Yates1991} is used. 
\begin{equation}
	r_{CO} = \frac{a p_{\mathrm{CO}}p_{\mathrm{H}_2}}  {\left(1+bp_{\mathrm{CO}} \right)^2}
\end{equation}
The original units are 
\begin{equation}
	\begin{split}
		a  	 	&[=] \frac{mmol}{min\,g_{cat}MPa^2}\\
		b 		&[=] \frac{1}{MPa}\\
		r_{CO} 	&[=] \frac{\frac{mmol}{min\,g_{cat}MPa^2} MPa\, MPa}{1 + \frac{1}{MPa} MPa}= \frac{mmol}{min\,g_{cat}} 
	\end{split}
\end{equation}
Converting to SI units gives
\begin{equation}
	\begin{split}
		a  	&[=] \frac{min}{60s}\frac{mol}{1000 mmol}\frac{kmol}{1000 mol}\frac{1000 g_{cat}}{kg_{cat}}\frac{mmol}{min\,g_{cat}MPa^2}\frac{MPa^2}{10^610^6Pa^2} \\
			& = \frac{10^{3}}{60\,10^3\,10^3\,10^6\,10^6}\frac{kmol}{s\,kg_{cat}\,Pa^2}\\
			& = \frac{10^{-15}}{60}\frac{kmol}{s\,kg_{cat}\,Pa^2}\\
		b 	&[=] \frac{1}{MPa}\\
			&= \frac{1}{MPa}\frac{MPa}{10^6Pa}\\
			&= \frac{10^{-6}}{Pa}\\
		r_{CO} &[=] \frac{\frac{10^{-15}}{60}\frac{kmol}{s\,kg_{cat}\,Pa^2}\,Pa Pa}{1+\frac{10^{-6}}{Pa}\,Pa} 
	\end{split}
\end{equation}

Furthermore, it is desired to express the components CO and H$_2$ in terms of their fractions in the liquid phase. We have from the assumptions that
\begin{equation}
	y_i^* = k_ix_i^*
\end{equation}
and can then write
\begin{equation}
	p_i^* = y_i^*p = k_ix_i^*p
\end{equation}

The final reaction rate expression is thus:
\begin{equation}
	\begin{split}
		r_{CO} = \frac{10^{-15}}{60} \frac{ak_{CO}x_{CO}^*p\,k_{H_2}x_{H_2}^*p}{1 + b\,10^{-6}\,k_{CO}x_{CO}^*p}\\
		r_{CO} = \frac{10^{-15}}{60} \frac{ak_{CO}x_{CO}^*\,k_{H_2}x_{H_2}^*p^2}{1 + b\,10^{-6}\,k_{CO}x_{CO}^*p}\\
	\end{split}
	\label{eq:final_reaction_rate_expression}
\end{equation}

\section{Reaction Term}
The reaction term from~\cite{Jakobsen08} was given as in equation~\ref{eq:reaction_term_jakobsen08}. 
\begin{equation}
	\epsilon_L \sum_r\nu_{i,r}R_{L,r} 
	\label{eq:reaction_term_jakobsen08}
\end{equation}
As the number of reactions in the chosen kinetic model is 1 (Yates and Satterfield (1991))~\cite{Yates1991}, we can remove the summation sign in~\ref{eq:reaction_term_jakobsen08} to obtain the expression given in~\ref{eq:reaction_term_jakobsen08_single_reaction}
\begin{equation}
	\epsilon_L \nu_{i}R_{L} 
	\label{eq:reaction_term_jakobsen08_single_reaction}
\end{equation}
The reaction rate expression $R_L$ taken from~\cite{Yates1991} and discussed in section~\ref{sec:kinetic_model} was finally obtained in equation~\ref{eq:final_reaction_rate_expression} in the units $r_{CO}[=]\frac{kmol}{s\, kg_{cat}}$. Studying equations~\ref{eq:component_balance_liquid_manipulations} and~\ref{eq:component_balance_liquid_minus_total_balance}, we see that the reaction term $\frac{\epsilon_L R_i}{\rho_L v_L^s}$overall shall have the units $\frac{1}{m}$. This requires the units $\epsilon_LR_i[=]\frac{kg \, m}{m^3\, s}\frac{1}{m}$. We see that this requires the following expression for $R_i$:

\begin{equation}
	\begin{split}
	[\epsilon_L R_i]&= \frac{kg \, m}{m^3\, s}\frac{1}{m} = [X][\epsilon_L\nu_ir_{CO}]= [X]\frac{kmol}{s\, kg_{cat}}\\
	[X]&=\frac{[\epsilon_L R_i]}{[\epsilon_L\nu_i r_{CO}]}= \frac{\frac{kg \, m}{m^3\, s}\frac{1}{m}}{\frac{kmol}{s\, kg_{cat}}} = \frac{kg}{kmol}\frac{kg_{cat}}{m^3}=[M_{w,i}][\rho_{cat}]
	\end{split}
\end{equation}
The final reaction rate term to be inserted into equation~\ref{eq:component_balance_liquid_minus_total_balance} is thus
\begin{equation}
	\frac{\epsilon_L \nu_i r_{CO}}{\rho_L v_L^s}M_{w,i}\rho_{cat}
\end{equation}

\section{Mass Transfer Term}
We have
\begin{equation}
	\begin{split}
		k_{L,i}a\rho_L(\omega_{i,L}^*-\omega_{i,L}) \\
	\end{split}
\end{equation}
The equilibrium criterium at the phase boundary between gas and liquid is given in equation~\ref{eq:GL_equilibrium_criterium}.
\begin{equation}
	y_i^* = k_ix_i^*
	\label{eq:GL_equilibrium_criterium}
\end{equation}
The mole fraction $x_i$ can be expressed in terms of mass fraction $\omega_i$ as given in equation~\ref{eq:mole_fraction_in_terms_of_mass_fraction}.
\begin{equation}
	x_i = \frac{\frac{\omega_i}{M_{w,i}}}{\sum_j \left( \frac{\omega_j}{M_{w,j}}\right)}
	\label{eq:mole_fraction_in_terms_of_mass_fraction}
\end{equation}
Inserting for $y_i^*$ and $x_i^*$ in equation~\ref{eq:GL_equilibrium_criterium} gives
\begin{equation}
	\frac{\frac{\omega_{i,G}^*}{M_{w,i}}}{\sum_j \left( \frac{\omega_{j,G}^*}{M_{w,j}}\right)} = k_i\frac{\frac{\omega_{i,L}^*}{M_{w,i}}}{\sum_j \left( \frac{\omega_{j,L}^*}{M_{w,j}}\right)} 
	\label{eq:GL_equilibrium_criterium_mass_fractions_inserted}
\end{equation}
Recognising the definition for average molar mass (equation~\ref{eq:average_molar_mass_mass_basis}) in terms of mass fractions, we can write the equilibrium expression as in equation~\ref{eq:GL_equilibrium_criterium_average_molar_mass_inserted}.
\begin{equation}
	\overline{M_{w}} = \sum_i x_i M_{w,i}= \sum_iM_{w,i}\frac{\frac{\omega_i}{M_{w,i}}}{\sum_j \left( \frac{\omega_j}{M_{w,j}}\right)} = \sum_i\frac{\omega_i}{\sum_j \left( \frac{\omega_j}{M_{w,j}}\right)}=\frac{1}{\sum_j \left( \frac{\omega_j}{M_{w,j}}\right)}
	\label{eq:average_molar_mass_mass_basis}
\end{equation}

\begin{equation}
	\begin{split}
		&\frac{\omega_{i,G}^*}{\sum_j \left( \frac{\omega_{j,G}^*}{M_{w,j}}\right)} = k_i\frac{\omega_{i,L}^*}{\sum_j \left( \frac{\omega_{j,L}^*}{M_{w,j}}\right)} \\
		&\overline{M_G}\omega_{i,G}^* = k_i \overline{M_L}\omega_{i,L}^*\\
		&...\\
		&\omega_{i,L}^* = \frac{1}{k_i}\frac{\overline{M_G}}{\overline{M_L}}\omega_{i,G}^* 
	\label{eq:GL_equilibrium_criterium_average_molar_mass_inserted}
	\end{split}
\end{equation}
The final form of the mass transfer terms is given in equation~\ref{eq:mass_transfer_term_final}.
\begin{equation}
	k_{L,i}a\rho_L(\frac{1}{k_i}\frac{\overline{M_G}}{\overline{M_L}}\omega_{i,G}^*-\omega_{i,L})
	\label{eq:mass_transfer_term_final}
\end{equation}

\subsection{Calculation of k$_L$ and a}
The correlation by Calderbank and Moo-Young ~\cite{CalderbankMoo-Young1961}, equation (2) for larger bubbles, was utilised:

\begin{equation}
		k_L = 0.42(N_{Sc})^{-1/2} \left(\frac{\Delta \rho \mu_c g}{\rho_c^2}\right)^{1/3}
\end{equation}
where $\Delta \rho = \rho_c - \rho_d$, i.e. density difference between continous (c) and dispersed (d) phase, and the dimensionless Schmidt number is defined as:
\begin{equation}
	Sc = \frac{\mu_c}{\rho_cD_c}
\end{equation}
Taking the liquid phase as the continous phase and the gas phase as the dispersed phase and adding the index "$i$" to denote component $i$, we get:
\begin{equation}
	\begin{split}
		k_{i,L} &= 0.42(N_{Sc})^{-1/2} \left(\frac{(\rho_L - \rho_G) \mu_L g}{\rho_L^2}\right)^{1/3} \\
		Sc &= \frac{\mu_L}{\rho_LD_{i,L}}
		\end{split}
\end{equation}
Diffusion coefficients for each of the components were calculated using the correlation by Erkey et al ~\cite{ErkeyRoddenAkgerman1990a}, table 5:
\begin{equation}
	10^9\frac{D_{12}}{\sqrt{T}} = \beta(V-V_D)
\end{equation}
where 
\begin{itemize}
	\item $D_{12}$ is the diffusion coefficient, $m^2/s$
	\item $T$ is the temperature, $T$
	\item $V$ is the molar volume of the solvent at the prevailing conditions, $cm^3/mol$
	\item $V_D$ is a parameter in the correlation, $cm^3/mol$
\end{itemize}
Further, 
\begin{equation}
	\begin{split}
	\beta &= \frac{94.5}{M_1^{0.239} M_2^{0.781}(\sigma_1\sigma_2)^{1.134}} \\
	V_D &= bV_0 \\
	V_0 &= \frac{N\sigma_2^3 }{\sqrt{2}}\\
	b &= 1.206 + 0.0632(\sigma_1 / \sigma_2) \\
	V = \frac{M_w}{\rho_L}
	\end{split}
\end{equation}
where (1) is the solute and (2) is the solvent. $\sigma$ is the hard sphere diameter, for which data were found in ~\cite{ErkeyRoddenAkgerman1990a} and supplemented with ~\cite{BirdStewartLightfoot2007} (lighter hydrocarbons) and ~\cite{Schatzberg1967} (water). 

Experimental results by the same authors in earlier work suggest that the complex wax mixture (solvent) can be treated as a pure n-alkane with the average carbon number of the wax ~\cite{ErkeyRoddenAkgerman1990b}. The average carbon number of the wax was 29. Octacosane was used as the solvent in their work and also here. The density of the solvent was also measured by ~\cite{ErkeyRoddenAkgerman1988} and thus used here.

The diffusion coefficients were reported to be only weakly dependent on pressure over the range of 0.1 - 3.40 MPa in previous work by the authors ~\cite{ErkeyRoddenAkgerman1988}. 

The resultant diffusion coefficients were well in line with those used by Maretto and Krishna ~\cite{MarettoKrishna1999} for $CO$ and $H_2$ (only those two components).


\section{Heat Transfer Terms}

\subsection{Heat transfer between slurry and gas}
The heat transfer between slurry and gas is given by
\begin{equation}
	h_L a (T_G - T_{SL})
\end{equation}

\subsection{Heat transfer between slurry and wall (cooling)}
The reaction is exothermal. Heat is removed by cooling tubes inside the reactor.

The heat transfer term denoting the cooling of the slurry by cooling tubes, $\frac{4U}{d_t}(T_{\mathrm{sur}}-T_{SL})$, must be re-written. The starting point from Jakobsen (2008)~\cite{Jakobsen08}, section 8.4, page 767 considers heating on the outside of the reactor perimeter only, but for Fischer-Tropsch synthesis an arrangment of multiple coolant tubes inside the reactor is more common. A more general form of the heat transfer term by cooling is given in Jakobsen (2008)~\cite{Jakobsen08}, equation 1.291:
\begin{equation}
	-h_W\frac{S_I}{A}(\langle T\rangle_A -\langle T \rangle_W)
\end{equation}
Considering a cooling arrangement with $N_t$ coolant tubes with the same length as the dispersion having a constant cross-sectional area $A_t$, we can write the perimeter $S_I$ and area $A$ of the cooling surface at each level of $z$ as:
\begin{equation}
	\begin{split}
		S_I &= 2\pi\frac{D_c}{2} + N_tS_t = \pi D_c + N_t\pi d_t = \pi \left(D_c + N_td_t \right)\\
		A &= A_{\mathrm{column}}-A_{\mathrm{cooling tubes}} = \pi \left(\frac{D_c}{2}\right)^2 - N_tA_t = \frac{\pi}{4}D_c^2 - N_t\frac{\pi}{4}d_t^2 = \frac{\pi}{4}\left(D_c^2-N_td_t^2\right)
	\end{split}
\end{equation}
This gives for the heat transfer term due to cooling of slurry:
\begin{equation}
	-h_W\frac{S_I}{A}(\langle T\rangle_A -\langle T \rangle_W) = h_W\frac{\pi \left(D_c + N_td_t \right)}{\frac{\pi}{4}\left(D_c^2-N_td_t^2\right)}(T_{\mathrm{surr}}-T_{SL}) =h_W \frac{D_c + N_td_t}{\frac{1}{4}(D_c^2-N_td_t^2)} (T_{\mathrm{surr}}-T_{SL})
\end{equation}


\section{Mass and Species Mass Conservation equations}

\subsection{Total mass conservation: gas phase}

For the total mass balance for the gas phase, equation~\ref{eq:total_mass_balance_gas}, the left hand side is expanded and the equation is manipulated to obtain an expression for the change in superficial gas velcocity by reactor length $\frac{d}{dz}v_G^s$ alone on the left hand side.
\begin{equation}
	\begin{split}
	\frac{d}{dz}\left(\rho_Gv_G^s\right) =-\rho_L\sum_ik_{L,i}a(\frac{1}{k_i}\frac{\overline{M_G}}{\overline{M_L}}\omega_{i,G}^*-\omega_{i,L})\\
	\rho_G\frac{d}{dz}v_G^s + v_G^s\frac{d}{dz}\rho_G =-\rho_L\sum_ik_{L,i}a(\frac{1}{k_i}\frac{\overline{M_G}}{\overline{M_L}}\omega_{i,G}^*-\omega_{i,L})\\
	\frac{d}{dz}v_G^s = -\frac{v_G^s}{\rho_G}\frac{d}{dz}\rho_G - \frac{\rho_L}{\rho_G}\sum_ik_{L,i}a(\frac{1}{k_i}\frac{\overline{M_G}}{\overline{M_L}}\omega_{i,G}^*-\omega_{i,L})
	\label{eq:total_balance_gas_manipulations}
	\end{split}
\end{equation}

\subsection{Species mass conservation: gas phase}
Starting out from the axial dispersion model, the species mass balance for the gas phase is split into two equations; one for flux and one for weight fractions.

We define:
\begin{equation}
	j_{i,G} = - \epsilon_G \rho_G D_{\mathrm{eff,z,G}}\frac{d\omega_{i,G}}{dz}
\end{equation}

We then get:
\begin{equation}
	\begin{split}
	& j_{i,G} + \epsilon_G \rho_G D_{\mathrm{eff,z,G}}\frac{d\omega_{i,G}}{dz} = 0\\
	&\frac{d}{dz}\left(\rho_G v_G^s \omega_{i,G} \right) + \frac{dj_{i,G}}{dz} + k_{L,i}a\rho_L(\frac{1}{k_i}\frac{\overline{M_L}}{\overline{M_G}}\omega_{i,G}^*-\omega_{i,L}) = 0
	\end{split}
\end{equation}
Finally, expanding the left hand side of the weight fraction equation.
	\begin{equation}
	\begin{split}
	&\rho_Gv_G^s\frac{d}{dz}\omega_{i,G} + \rho_G\omega_{i,G} \frac{d}{dz}v_G^s + v_G^s \omega_{i,G} \frac{d}{dz}\rho_G + \frac{dj_{i,G}}{dz} + k_{L,i}a\rho_L(\frac{1}{k_i}\frac{\overline{M_L}}{\overline{M_G}}\omega_{i,G}^*-\omega_{i,L}) = 0
	\end{split}
\end{equation}


\subsection{Total mass conservation: liquid phase}

For the total mass balance in the liquid phase, we insert for the mass transfer term (equation~\ref{eq:mass_transfer_term_final}) into equation~\ref{eq:total_mass_balance_liquid}. Further, we expand the left hand side and obtain an expression for $\frac{d}{dz}v_L^s$.
\begin{equation}
	\begin{split}
	&\frac{d}{dz}\left(\rho_Lv_L^s\right) =\rho_L\sum_ik_{L,i}a(\frac{1}{k_i}\frac{\overline{M_G}}{\overline{M_L}}\omega_{i,G}^*-\omega_{i,L}) \\
	&\rho_L\frac{d}{dz}v_L^s + v_L^s\frac{d}{dz}\rho_L = \rho_L\sum_ik_{L,i}a(\frac{1}{k_i}\frac{\overline{M_G}}{\overline{M_L}}\omega_{i,G}^*-\omega_{i,L}) \\
	&\frac{d}{dz}v_L^s = -\frac{v_L^s}{\rho_L}\frac{d}{dz}\rho_L + \sum_ik_{L,i}a(\frac{1}{k_i}\frac{\overline{M_G}}{\overline{M_L}}\omega_{i,G}^*-\omega_{i,L})
	\label{eq:total_mass_balance_liquid_manipulations}
	\end{split}
\end{equation}

\subsection{Species mass conservation: liquid phase}
Starting out from the axial dispersion model, the species mass balance for the liquid phase is split into two equations; one for flux and one for weight fractions.

We define:
\begin{equation}
	j_{i,L} = - \epsilon_L \rho_L D_{\mathrm{eff,z,L}}\frac{d\omega_{i,L}}{dz}
\end{equation}

We then get:
\begin{equation}
	\begin{split}
	& j_{i,L} + \epsilon_L \rho_L D_{\mathrm{eff,z,L}}\frac{d\omega_{i,L}}{dz} = 0\\
	&\frac{d}{dz}\left(\rho_L v_L^s \omega_{i,L} \right) + \frac{dj_{i,L}}{dz} - k_{L,i}a\rho_L(\frac{1}{k_i}\frac{\overline{M_L}}{\overline{M_G}}\omega_{i,G}^*-\omega_{i,L}) = \epsilon_LR_i 
	\end{split}
\end{equation}
Finally, expanding the left hand side of the weight fraction equation.
	\begin{equation}
	\begin{split}
	&\rho_Lv_L^s\frac{d}{dz}\omega_{i,L} + \rho_L\omega_{i,L} \frac{d}{dz}v_L^s + v_L^s \omega_{i,L} \frac{d}{dz}\rho_L + \frac{dj_{i,L}}{dz} - k_{L,i}a\rho_L(\frac{1}{k_i}\frac{\overline{M_L}}{\overline{M_G}}\omega_{i,G}^*-\omega_{i,L}) = \epsilon_LR_i 
	\end{split}
\end{equation}

\subsection{Rate of Change in Volume Fraction}
The rate of change in volume fraction is set to zero at the current model stage. This will be improved. The volume fraction of gas is estimated, see separate note.

\subsection{Rate of Change in Density}
Equations~\ref{eq:component_balance_gas_manipulations} and~\ref{eq:total_balance_gas_manipulations} require an expression for the rate of change in gas density, $\frac{d}{dz}\rho_G$.

The ideal gas law (~\ref{eq:ideal_gas_law}) was assumed to be valid in section~\ref{sec:assumptions}. An expression for gas density given the ideal gas law valid is given in 

\begin{equation}
	pV = nRT
	\label{eq:ideal_gas_law}
\end{equation}

\begin{equation}
	\rho_G=\frac{n}{V}\overline{M_G}=\frac{p\overline{M_G}}{RT}
	\label{eq:gas_density_by_ideal_gas_law}
\end{equation}
This yields the expression shown in equation~\ref{eq:gas_density_rate_of_change}.
\begin{equation}
	\frac{d}{dz}\rho_G = \frac{d}{dz}\left( \frac{p\overline{M_G}}{RT}\right) = \frac{p}{RT}\frac{d}{dz} \overline{M_G} + \frac{\overline{M_G} }{RT}\frac{dp}{dz} - \frac{p\overline{M_G}}{R}\frac{\frac{dT_G}{dz}}{T_G^2}
	\label{eq:gas_density_rate_of_change}
\end{equation}
Equation~\ref{eq:gas_density_rate_of_change} requires an expression for the rate of change in average molar mass for the gas phase $\frac{d}{dz}\overline{M_G}$ and for the pressure $\frac{dp}{dz}$. These are derived in sections~\ref{sec:rate_of_change_average_molar_mass} and ~\ref{sec:rate_of_change_pressure}, respectively.

\subsection{Rate of Change in Pressure}
\label{sec:rate_of_change_pressure}
A hydrostatic pressure profile is applied;
\begin{equation}
	p(z) = \int_0^L\rho_{\mathrm{tot}}(z)gdz
\end{equation}
The total density $\rho_{\mathrm{tot}}$ is assumed constant and given by
\begin{equation}
	\rho_{\mathrm{tot}} = \epsilon_G^0\rho_G^0 + \epsilon_L^0\rho_L + \epsilon_s\rho_s
\end{equation}
Inserting for  $\rho_{\mathrm{tot}}(z)=\rho_{\mathrm{tot}}$ and utilising $h = L-z$
\begin{equation}
	\begin{split}
	p(z) &= \int_0^h\rho_{\mathrm{tot}}gdz\\
	p(z) &= \rho_{\mathrm{tot}}gh \\
	p(z) &= \rho_{\mathrm{tot}}g(L-z)
	\end{split}
\end{equation}
Differentiating with respect to $z$ then gives
\begin{equation}
	\frac{dp}{dz} = -\rho_{\mathrm{tot}}g
	\label{eq:rate_of_change_pressure}
\end{equation}

\subsection{Rate of Change in Average Molar Mass}
\label{sec:rate_of_change_average_molar_mass}
Equation~\ref{eq:average_molar_mass_mass_basis} provides us with an expression for $\overline{M_G}$, which we insert into equation~\ref{eq:gas_density_rate_of_change} and differentiate. The result is given below. 

\begin{equation}
	\begin{split}
	\frac{d}{dz} \overline{M_G} &=  \frac{d}{dz}\left(\sum_ix_iM_{w,i}\right) = \sum_i M_{w,i} \frac{d}{dz}\left( \frac{\frac{\omega_i^G}{M_{w,i}}}{\sum_j\frac{\omega_j^G}{M_{w,j}}}\right)\\
	&= \sum_i M_{w,i} \left[\frac{\omega_i^G}{M_{w,i}}\frac{d}{dz}\left(\frac{1}{\sum_j \frac{\omega_j^G}{M_{w,j}}} \right) + \frac{1}{\sum_j \frac{\omega_j^G}{M_{w,j}}}  \frac{1}{M_{w,i}}\frac{d \omega_i^G}{dz}\right] \\
	&= \sum_i \left[\omega_i^G\frac{d}{dz}\left(\frac{1}{\sum_j \frac{\omega_j^G}{M_{w,j}}} \right) + \frac{1}{\sum_j \frac{\omega_j^G}{M_{w,j}}}\frac{d \omega_i^G}{dz}\right] \\
	&= \sum_i \left[\omega_i^G\left(-\frac{1}{\left(\sum_j \frac{\omega_j^G}{M_{w,j}}\right)^2} \frac{d}{dz}\left[\sum_j \frac{\omega_j^G}{M_{w,j}} \right]  \right)  + \frac{1}{\sum_j \frac{\omega_j^G}{M_{w,j}}}\frac{d \omega_i^G}{dz}\right] \\
	&= \sum_i \left[\omega_i^G\left(-\frac{1}{\left(\sum_j \frac{\omega_j^G}{M_{w,j}}\right)^2} \left[ \sum_j \frac{\omega_j^G}{M_{w,j}}\frac{d \omega_j^G}{dz} \right]  \right)  + \frac{1}{\sum_j \frac{\omega_j^G}{M_{w,j}}}\frac{d \omega_i^G}{dz}\right] \\
	&= \sum_i \left[\omega_i^G\left(-\frac{1}{\left(\sum_j \frac{\omega_j^G}{M_{w,j}}\right)^2} \left[ \sum_j \frac{\omega_j^G}{M_{w,j}}\frac{d \omega_j^G}{dz} \right]  \right)  + \frac{1}{\sum_j \frac{\omega_j^G}{M_{w,j}}}\frac{d \omega_i^G}{dz}\right] \\
	&= \sum_i \left[\frac{1}{\sum_j \frac{\omega_j^G}{M_{w,j}}}\frac{d \omega_i^G}{dz} - \omega_i^G\frac{\sum_j \frac{1}{M_{w,j}}\frac{d \omega_j^G}{dz}}{\left(\sum_j \frac{\omega_j^G}{M_{w,j}}\right)^2}\right] \\
	&= \sum_i \left[\frac{\frac{d \omega_i^G}{dz}\sum_j \frac{\omega_j^G}{M_{w,j}} - \omega_i^G \sum_j \frac{\frac{d \omega_j^G}{dz}}{M_{w,j}}}{\left(\sum_j \frac{\omega_j^G}{M_{w,j}}\right)^2}\right] \\
	\frac{d}{dz} \overline{M_G} &= \frac{1}{\left(\sum_j \frac{\omega_j^G}{M_{w,j}}\right)^2}\sum_i \left[\frac{d \omega_i^G}{dz}\sum_j \frac{\omega_j^G}{M_{w,j}} - \omega_i^G \sum_j \frac{d \omega_j^G}{dz}\frac{1}{M_{w,j}}\right]
	\end{split}
\end{equation}

\section{Energy Conservation}
The starting point for the temperature equations was given in equations~\ref{eq:energy_balance_gas} and~\ref{eq:energy_balance_liquid}. As stated in the section of Assumptions,~\ref{sec:assumptions}, the thermal conductivity coefficient is constant for both phases. The equations were re-written into a set of two equations per phase; one heat flux and one temperature equation.

\subsection{Temperature equation: gas phase}
\begin{equation}
	\rho_G C_{p,G} v_G^s\frac{dT_G}{dz} = \frac{d}{dz}\left(\epsilon_G \lambda_{\mathrm{eff,z,G}}\frac{dT_G}{dz} \right) - h_L a (T_G - T_{SL})
\end{equation}
We define:
\begin{equation}
	q_G = - \epsilon_G \lambda_{\mathrm{eff,z,G}}\frac{dT_G}{dz} 
\end{equation}
This yields:
\begin{equation}
	\begin{split}
		&q_G + \epsilon_G \lambda_{\mathrm{eff,z,G}}\frac{dT_G}{dz} = 0 \\
		&\rho_G C_{p,G} v_G^s\frac{dT_G}{dz} + \frac{dq_G}{dz}- h_L a (T_G - T_{SL}) = 0
	\end{split}
\end{equation}

\subsection{Temperature equation: slurry phase}
\begin{equation}
	\rho_{SL} C_{p,SL} v_{SL}^s\frac{d T_{SL}}{dz} = \frac{d}{dz}\left(\epsilon_{SL}\lambda_{\mathrm{eff,z,L}}\frac{d T_{SL}}{dz}\right) + h_W\frac{S_I}{A}(T_{\mathrm{surr}}-T_{SL}) + h_L a (T_G - T_{SL}) + \epsilon_{L}(-\Delta H_r)r_{CO}\rho_{\mathrm{cat}}
\end{equation}
We define:
\begin{equation}
	q_{SL} = - \epsilon_{SL} \lambda_{\mathrm{eff,z,SL}}\frac{dT_{SL}}{dz}
\end{equation}
This gives:
\begin{equation}
	\begin{split}
	&q_{SL} + \epsilon_{SL} \lambda_{\mathrm{eff,z,SL}}\frac{dT_{SL}}{dz} = 0\\
	&\rho_{SL} C_{p,SL} v_{SL}^s\frac{d T_{SL}}{dz} + \frac{d q_{SL}}{dz} - h_W\frac{S_I}{A}(T_{\mathrm{surr}}-T_{SL}) - h_L a (T_G - T_{SL}) = \epsilon_{L}(-\Delta H_r)r_{CO}\rho_{\mathrm{cat}}
	\end{split}
\end{equation}

\section{Parameters}
See separate note on the non-adjustable operating conditions.

\section{Final Model Equations}

\begin{equation}
	\epsilon_L + \epsilon_G + \epsilon_S = 1
\end{equation}
\begin{equation}
	\begin{split}
	\frac{d \epsilon_L}{dz} &= 0 \\
	\frac{d \epsilon_S}{dz} &= 0 \\
	\frac{d \epsilon_G}{dz} &= 0
	\end{split}
\end{equation}

\begin{equation}
	\begin{split}
		\frac{d \rho_G}{dz} &= \frac{p}{RT}\frac{d\overline{M_G}}{dz} + \frac{\overline{M_G}}{RT}\frac{dp}{dz} - \frac{\overline{M_G}p \frac{dT_G}{dz}}{R T_G^2} \\
		\frac{d \rho_L}{dz} &= 0
	\end{split}
\end{equation}

\begin{equation}
	\begin{split}
		\frac{d \overline{M_G}}{dz} &= \frac{1}{\left(\sum_j \frac{\omega_j^G}{M_{w,j}}\right)^2}\sum_i\left[\frac{d\omega_{G,i}}{dz} \sum_j\frac{\omega_j^G}{M_{w,j}} - \omega_{i,G}\sum_j \frac{d\omega_{G,j}}{dz}\frac{1}{M_{w,j}} \right] \\
		\frac{d \overline{M_L}}{dz} &= \frac{1}{\left(\sum_j \frac{\omega_j^L}{M_{w,j}}\right)^2}\sum_i\left[\frac{d\omega_{L,i}}{dz} \sum_j\frac{\omega_j^L}{M_{w,j}} - \omega_{i,L}\sum_j \frac{d\omega_{L,j}}{dz}\frac{1}{M_{w,j}} \right] \\
	\end{split}
\end{equation}
Note! The rate of change in molar mass was implemented as follows:
\begin{equation}
	\frac{d \overline{M_G}}{dz} = D_{ij}\overline{M_G}
\end{equation}
where $D_{ij}$ is the derivative matrix of the interpolation polynomials. 

\begin{equation}
	\begin{split}
		\frac{dv_G^s}{dz} &= -\frac{v_G^s}{\rho_G}\frac{d}{dz}\rho_G - \frac{\rho_L}{\rho_G}\sum_ik_{L,i}a(\frac{1}{k_i}\frac{\overline{M_G}}{\overline{M_L}}\omega_{i,G}^*-\omega_{i,L}^*)\\
		\frac{dv_L^s}{dz} &= -\frac{v_L^s}{\rho_L}\frac{d}{dz}\rho_L + \sum_ik_{L,i}a(\frac{1}{k_i}\frac{\overline{M_G}}{\overline{M_L}}\omega_{i,G}^*-\omega_{i,L}^*)
	\end{split}
\end{equation}

Species mass balance: dispersion flux and weight fraction equations.\\
Gas phase:
\begin{equation}
	\begin{split}
	& j_{i,G} + \epsilon_G \rho_G D_{\mathrm{eff,z,G}} \frac{d\omega_{i,G}}{dz} = 0\\
	& \rho_Gv_G^s \frac{d\rho_G}{dz} + \rho_Gv_G^s\frac{d\omega_{i,G}}{dz} + \rho_G\omega_{i,G}\frac{dv_G^s}{dz} + \frac{dj_{i,G}}{dz} + k_{L,i}a(\frac{1}{k_i}\frac{\overline{M_G}}{\overline{M_L}}\omega_{i,G}^*-\omega_{i,L}) = 0
	\end{split}
\end{equation}
Liquid phase:
\begin{equation}
	\begin{split}
	& j_{i,L} + \epsilon_L \rho_L D_{\mathrm{eff,z,L}} \frac{d\omega_{i,L}}{dz} = 0\\
	&\rho_Lv_L^s \frac{d\rho_L}{dz} + \rho_Lv_L^s\frac{d\omega_{i,L}}{dz} + \rho_L\omega_{i,L}\frac{dv_L^s}{dz} + \frac{dj_{i,L}}{dz} - k_{L,i}a(\frac{1}{k_i}\frac{\overline{M_G}}{\overline{M_L}}\omega_{i,G}^*-\omega_{i,L})= \epsilon_L \nu_i r_{CO} M_{w,i}\rho_{cat}
	\end{split}
\end{equation}

Energy balance: heat flux and temperature equations.\\
Gas phase:
\begin{equation}
\begin{split}
& \frac{q_G}{\epsilon_G\lambda_{\mathrm{eff,z,G}}} + \frac{dT_G}{dz} = 0\\
& \frac{dT_G}{dz} + \frac{1}{\rho_G C_{p,G} v_G^s}\frac{dq_G}{dz} + \frac{h_L a}{\rho_G C_{p,G} v_G^s}(T_G - T_{SL}) = 0
\end{split}
\end{equation}
Liquid phase:
\begin{equation}
\begin{split}
& \frac{q_{SL}}{\epsilon_{SL}\lambda_{\mathrm{eff,z,SL}}} + \frac{dT_{SL}}{dz} = 0 \\
& \frac{d T_{SL}}{dz} + \frac{1}{\rho_{SL}C_{p,SL}v_{SL}^s}\frac{dq_{SL}}{dz} - h_W\frac{S_I}{A}\frac{(T_{\mathrm{surr}}-T_{SL})}{\rho_{SL}C_{p,SL}v_{SL}^s} - \frac{h_L a(T_G - T_{SL})}{\rho_{SL}C_{p,SL}v_{SL}^s} = \frac{\epsilon_{L}\sum_r (-\Delta H_r)r_{CO}\rho_{\mathrm{cat}}}{\rho_{SL}C_{p,SL}v_{SL}^s}
\end{split}
\end{equation}

\bibliography{M:/literature/cbv}
\bibliographystyle{plain} 



\end{document}